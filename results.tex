\documentclass[12pt]{article}
\usepackage[utf8]{inputenc}
\usepackage{amsmath}
\usepackage{amsfonts}
\usepackage{amssymb}
\usepackage{multirow}
\usepackage{siunitx}
\usepackage{geometry}
\usepackage{pdflscape}
\usepackage{rotating}
\usepackage[plain]{fancyref}
\usepackage{hyperref}
\hypersetup{
    colorlinks=true,
    linkcolor=blue,
    urlcolor=blue
}
\begin{document}

\title{2019-10-06 Cupping session}

\tableofcontents

\section{Data entry and handling} \label{sec:Data}

Five coffees were tested (with urls!), with 5 different tasters:
\begin{itemize}
	\item \href{https://www.intelligentsiacoffee.com/zirikana-rwanda}{Intelligentsia's Zirikana Rwanda~(ZR)}
	\item \href{https://www.jbccoffeeroasters.com/tano-batak-sumatra}{JBC's Tano Batak Sumatra~(JBC)}
	\item \href{https://bluebottlecoffee.com/store/beta-blend}{Blue Bottle's Beta Blend~(BBBB)}
	\item \href{https://www.target.com/p/organic-ethiopian-yirgacheffe-light-roast-whole-bean-coffee-10oz-archer-farms-153/-/A-50566950}{Archer Farms' Ethiopian Organic~(AF)}
	\item \href{https://rubycoffeeroasters.com/collections/coffee/products/ethiopia-benti-nenka}{Ruby's Benti Nenka~(Ruby)}
\end{itemize}

I converted each scale on the form into a value from $0$ to $4$.
No, I don't know why I didn't use $1$ to $5$.
I roughly estimated where to enter values when they were halfway between data points, this probably skewed the data slightly because I was very lazy and imprecise.
The summary statistics for the data are presented in \fref{tab:statistics}, with averages and standard deviations.
In \fref{tab:statistics}, the ``Q'' column includes our data for the quality of the characteristic (as in, how much we enjoyed or disliked it), in contrast to the ``I'' column, which represents intensity (how strong or weak it was).

A list of the descriptors used is provided in \fref{tab:descriptors}.
I changed some spellings (some people wrote flavor instead of flavour omg), but otherwise just threw them in a list.
Any duplicates in that table are because two people each commented the same thing.

\section{Salient features from the data} \label{sec:SalientFeatures}
Some interesting notes (although I did zero statistical analysis, so most may not actually be true):
\begin{itemize}
	\item Our standard deviations were all smaller than $2$, and usually around $1$, which means we were consistently able to narrow down a value to at least half the range (which I count as a win).
	\item The standard deviations for aroma, body and finish were all somewhat larger than for sweetness and acidity.
	I'm guessing this was random chance and means nothing.
	\item ZR was billed as sweet and tart, but its acidity and sweetness intensities were very comparable to the other coffees being compared.
	JBC had a higher sweetness intensity, and Ruby had a higher acidity intensity.
	We enjoyed the tartness of the ZR more than we enjoyed any other intensity.
	\item BBBB was a blend, which theoretically should leave it smoother and less distinctive.
	We largely detected this effect: its intensity scores were in the middle of the pack (excluding Target), and we noticed it having a very short finish and very slight body.
	Even more telling, the descriptors for this are striking, picking it out as light, not very distinct and very coffee-esque.
	That's impressive
	\item Our statistics for AF are actually pretty scattergun, especially for aroma and acidity.
	My theory is that its odd taste was hard for us to characterise, so we couldn't tell if the bad part of the flavour came from acidity or not.
	It was by far the least sweet, which could have been because any sweetness it had was drenched by bitterness.
	I believe it's fair to say that we were able to identify it as the Target coffee, even through the double-blinding procedure.
\end{itemize}

\section{Overall score} \label{sec:Overall}
Overall scores are presented in \fref{tab:overall}.
The overall column represents the scores entered on the forms.
Because not all of the surveys included very granular data, I mapped ranking systems to point values and created the ``Calculated Score'' column, which also incorporates the aggregate calculations some of us did.
I'd interpret the ``Calculated Score'' column as my personal interpretation of other people's overall scores, rather than anything rigorous, meaningful, or reflective of anyone else's opinions.
It only occurs in \fref{tab:statistics} because I was very proud of getting the columns to line up properly even for the values spanning multiple rows, and refused to remove it when I realised it would be unnecessary.

Overall, it's fair to say we had a least-favourite.
Ruby wasn't at the top of every list, but it scored very high overall, and as a result had the highest average scores
Interestingly, it received very similar scores to ZR for quality in the different measures,  which comes in second place.
I might conjecture that BBBB over-performed in the overall score vs the calculated one because it was broadly inoffensive, but indistinct.
As a result, no individual quality was spectacular (every other non-AF coffee had higher quality scores than BBBB's peak quality in sweetness).
It seems like for the JBC coffee, it had a more pronounced flavour, with more of every characteristic besides acidity than BBBB.
Our descriptors for JBC reflected the stronger taste, but that intensity wasn't necessarily entirely pleasant.

\section{Conclusion} \label{sec:Conclusions}
I think the main takeaways are as follows:
\begin{itemize}
	\item We were all able to distinguish good coffee from Target
	\item We all noted the lightness and sweetness of the Ruby Benti Nenka
	\item Our aggregate data was able to pick out a blend
	\item Baklava are delicious
\end{itemize}

\appendix
\section{Tables} \label{sec:tables}
\newgeometry{margin=2.6cm} % modify this if you need even more space
\begin{landscape}
	\begin{table}
		\begin{tabular}{@{} r||c|S[table-format=3.2(4),separate-uncertainty]|S[table-format=3.2(4),separate-uncertainty]|S[table-format=3.2(4),separate-uncertainty]|S[table-format=3.2(4),separate-uncertainty]|S[table-format=3.2(4),separate-uncertainty]| c @{}}
			\multicolumn{1}{c||}{Coffee} &  & \multicolumn{1}{|c|}{Aroma} & \multicolumn{1}{|c|}{Acidity }& \multicolumn{1}{|c|}{Sweetness} & \multicolumn{1}{|c|}{Body} & \multicolumn{1}{|c|}{Finish} & {Calculated Score}\\
			\hline
			\multirow{2}{5em}{ZR} & Q & 2.05 \pm 1.22 & 3.73 \pm 0.55 & 2.73 \pm 0.49 & 3 \pm 0.82 & 2.45 \pm 0.97 & \multirow{2}*{\tablenum[separate-uncertainty]{6.38 \pm 0.85}} \\
			& I & 1.3 \pm 1.25 & 2.5 \pm 0.87 & 2.12 \pm 0.76 & 2.5 \pm 1.32 & 1.75 \pm 0.96 \\
			\hline
			\multirow{2}{5em}{JBC}  & Q & 2 \pm 1.15 & 2.73 \pm 0.95 & 3 \pm 0.82 & 2.25 \pm 0.5 & 2.1 \pm 0.99 &\multirow{2}*{\tablenum[separate-uncertainty]{6.13 \pm 1.31}} \\
			& I & 2.05 \pm 0.75 & 1.88 \pm 1.25 & 2.68 \pm 0.46 & 1.68 \pm 0.46 & 2 \pm 1.41 \\
			\hline
			\multirow{2}{5em}{BBBB} & Q & 1.5 \pm 1 & 1.75 \pm 0.5 & 2.45 \pm 1.29 & 2 \pm 0.82 & 2 \pm 1.41 &\multirow{2}*{\tablenum[separate-uncertainty]{5.13 \pm 0.85}} \\
			& I & 1.3 \pm 1.01 & 2.24 \pm 1.03 & 2.14 \pm 0.67 & 0.6 \pm 0.89 & 0.65 \pm 0.47 \\
			\hline
			\multirow{2}{5em}{AF} & Q & 0.25 \pm 0.5 & 1.63 \pm 1.11 & 0.55 \pm 0.64 & 0.88 \pm 1.03 & 0.5 \pm 0.58 &\multirow{2}*{\tablenum[separate-uncertainty]{1.75 \pm 0.5}} \\
			& I & 2.5 \pm 1.91 & 2.32 \pm 1.42 & 1.2 \pm 0.84 & 1.8 \pm 0.84 & 3.25 \pm 0.96 \\
			\hline
			\multirow{2}{5em}{Ruby} & Q & 3.35 \pm 0.94 & 3.2 \pm 0.91 & 3.03 \pm 1.42 & 3.15 \pm 1.01 & 2.88 \pm 1.31 &\multirow{2}*{\tablenum[separate-uncertainty]{8.25 \pm 2.36}} \\
			& I & 1.98 \pm 1.07 & 2.74 \pm 1.02 & 2.16 \pm 1.46 & 1.3 \pm 1.54 & 2.25 \pm 0.96 \\
			\hline
		\end{tabular}
		\caption{Summary statistics from data, with standard deviation}
		\label{tab:statistics}
	\end{table}

	\begin{table}
		\begin{tabular}{c || c}
			ZR & earthy, morning, roasty, wood, fruity\\
			\hline
			JBC & animal, celery dipped in saltwater, fruity, nice intensity, agave, chocolatey \\
			\hline
			BBBB & earthy, medical, the coffiest, not a ton of distinct flavour, coffeey, fruity, not as heavy \\
			\hline
			AF & nuts, medical, bitter, sour, like I ate something I wasn't supposed to, meh, bitter, turpentine, most normal, is this target \\
			\hline
			Ruby & floral, chocolate, round/spherical, blue-grey, nutty, mild, lychee, yay! \\
			\hline
		\end{tabular}
		\caption{Descriptors and taste notes}
		\label{tab:descriptors}
	\end{table}

	\begin{table}
		\begin{tabular}{c || S[table-format=3.2(4),separate-uncertainty] | S[table-format=3.2(4),separate-uncertainty]}
			{Coffee} & {Overall Score} & {Calculated Score} \\
			\hline
			ZR & 7.00 \pm 0.87 & 6.38 \pm 0.85 \\
			\hline
			JBC &  5.00 \pm 1.00 & 6.13 \pm 1.31 \\
			\hline
			BBBB & 6.17 \pm 1.61 & 5.13 \pm 0.85 \\
			\hline
			AF & 1.67 \pm 0.58 & 1.75 \pm 0.5 \\
			\hline
			Ruby & 8.00 \pm 1.00 & 8.25 \pm 2.36 \\
			\hline
		\end{tabular}
		\caption{Overall scores (with description for the calculated column in the text)}
		\label{tab:overall}
	\end{table}
\end{landscape}
\restoregeometry
\end{document}